\documentclass[a4paper]{article}

%% Language and font encodings
\usepackage[english]{babel}
\usepackage[utf8x]{inputenc}
\usepackage[T1]{fontenc}
\usepackage{algpseudocode} 
\usepackage{float}
\usepackage{booktabs}
\usepackage{amsmath}
\newtheorem{theorem}{Theorem}
%% Sets page size and margins
\usepackage[a4paper,top=3cm,bottom=2cm,left=3cm,right=3cm,marginparwidth=1.75cm]{geometry}

%% Useful packages
\usepackage{amsmath}
\usepackage{graphicx}
\usepackage{qtree}
\usepackage[colorinlistoftodos]{todonotes}
\usepackage[colorlinks=true, allcolors=blue]{hyperref}
\usepackage{tikz}
\usetikzlibrary{automata,positioning}
\usepackage{amsfonts}
\usepackage{amssymb}

\usepackage{forest}
\renewcommand{\rmdefault}{ptm}

\begin{document}
\section*{Family of uniform hash functions}
The notion of pairwise independence says that, for any $x1 \neq x2$ and $c1, c2 \in Z_p$, we have that
\[
\underset{h \in \mathcal{H}}{Pr}[h(x_1) = c_1 \wedge h(x_2) = c_2] = \underset{h \in \mathcal{H}}{Pr}[h(x_1) = c_1] \times \underset{h \in \mathcal{H}}{Pr}[h(x_2) = c_2] 
\]\qquad \\
In other words, the joint probability is the product of the two individual probabilities. Show that the family of hash functions $H = \{ h_{ab}(x) = ((ax + b) \ mod \ p)\ mod \ m \ : a \in Z^*_p, b \in Z_p\}$ (seen in class) is "pairwise independent", where $p$ is a sufficiently large prime number $(m + 1 \leq p \leq 2m)$.
\\
\\
\textbf{SOLUTION}
\\
Each probability from the right hand side of the equation are equal to
\(\frac{1}{m}\), simply because there are \(m\) buckets chosen uniformly (Universal hash function).
Therefore, in the right hand side
we have a probability equal to \(\frac{1}{m^2}\). On the left side of the equation, \(h(x_1)\) 
and \(h(x_2)\) can be rewrite with their expression, such as:
\[
\left.
\begin{array}{l}
\text{$h(x_1)$ = ($a \cdot x_1 + b$ mod $p$) mod $m$}\\
\text{$h(x_2)$ = ($a \cdot x_2 + b$ mod $p$) mod $m$}
\end{array}
\right. \Bigl\}
x_1 \ne x_2
\]
\qquad \\
Now, let's consider the inner modulo \(p\) expressions as \(r = a\cdot x_1 + b\) mod
\(p\) and \(s = a \cdot x_2 + b\) mod \(p\), since $p$ is prime and there is a linear transformation). 
Thus we can rewrite the original expression as: \({Pr}[r \text{ mod } m = c_1 \wedge s \text{ mod } m = c_2]\). Since
$a \in [1 \dots p-1]=\mathbb{Z}^*$ and $b \in [1 \dots p]=\mathbb{Z}$, and both \(a\) and \(b\) are fixed for \(r\) and \(s\), then there are \(p\cdot(p-1)\) possible combination. \\
\qquad \\
Now, we focus on \(c_1\) and \(c_2\). When \(p\) is sufficiently
larger than \(m\), the number of values that will match with a given
value \(c_i\) are \(\frac{p}{m}\), as we have seen in class. 
However, we need to take in account modular arithmetic: not all values of \(m\)
is going to be a multiple of \(p\). Adding this into the
analysis, the number of matches are bound by considering the
round-to-floor and round-to-ceil limits:

\[
\Bigl \lfloor \frac{p}{m} \Bigr \rfloor \le \textsc{\# of values matching a value} c_i \le \Bigl \lceil \frac{p}{m} \Bigr\rceil
\]
\qquad \\
Since we have $c_1$ and $c_2$, we need to consider their joint probability (i.g. multiplication), and then we can be bound it by combining both pieces of the analysis (dividing bad cases by all possible combination), giving us an approximation to the expected $\frac{1}{m^2}$:

$$
\frac{1}{p\cdot(p-1)}\Bigl(\Bigl\lfloor\frac{p}{m}\Bigr\rfloor\Bigr)^2  \le {Pr}[r \text{ mod } m = c_1 \wedge s \text{ mod } m  = c_2] \le \frac{1}{p\cdot(p-1)}\Bigl(\Bigl\lceil\frac{p}{m}\Bigr\rceil\Bigr)^2 
$$
\qquad \\
It must be noted that the $\frac{1}{m^2}$ result may not be fully reached since that all the possible values $a$ are $p-1$, which make it impossible to remove the $\frac{p}{p-1}$ term. The only possibility is when $a = 0$ that would not make any sense algorithmically speaking. Therefore the result can only be approximated.

\end{document}