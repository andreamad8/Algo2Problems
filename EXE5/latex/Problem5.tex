\documentclass[a4paper]{article}

%% Language and font encodings
\usepackage[english]{babel}
\usepackage[utf8x]{inputenc}
\usepackage[T1]{fontenc}
\usepackage{algpseudocode} 
\usepackage{float}
\usepackage{booktabs}
\usepackage{amsmath}
\newtheorem{theorem}{Theorem}
%% Sets page size and margins
\usepackage[a4paper,top=3cm,bottom=2cm,left=3cm,right=3cm,marginparwidth=1.75cm]{geometry}

%% Useful packages
\usepackage{amsmath}
\usepackage{graphicx}
\usepackage{qtree}
\usepackage[colorinlistoftodos]{todonotes}
\usepackage[colorlinks=true, allcolors=blue]{hyperref}
\usepackage{tikz}
\usetikzlibrary{automata,positioning}

\usepackage{forest}
\renewcommand{\rmdefault}{ptm}

\begin{document}
\section*{Family of uniform hash functions}
The notion of pairwise independence says that, for any $x1 \neq x2$ and $c1, c2 \in Z_p$, we have that
\begin{equation}
Pr_{h\in H}[h(x1) = c1  \land  h(x2) = c2] = Pr_{h\in H} [h(x1) = c1] * Pr_{h\in H}[h(x2) = c2]
\nonumber
\end{equation}
In other words, the joint probability is the product of the two individual probabilities. Show that the family of hash functions $H = \{ h_{ab}(x) = ((ax + b) \ mod \ p)\ mod \ m \ : a \in Z^*_p, b \in Z_p\}$ (seen in class) is "pairwise independent", where $p$ is a sufficiently large prime number $(m + 1 \leq p \leq 2m)$.
\\
\\
\textbf{SOLUTION}
\\

\end{document}