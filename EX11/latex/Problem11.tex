\documentclass[a4paper]{article}

%% Language and font encodings
\usepackage[english]{babel}
\usepackage[utf8x]{inputenc}
\usepackage[T1]{fontenc}
\usepackage{algpseudocode}
\usepackage{float}
\usepackage{booktabs}
\usepackage{amsmath}
\newtheorem{theorem}{Theorem}
%% Sets page size and margins
\usepackage[a4paper,top=3cm,bottom=2cm,left=3cm,right=3cm,marginparwidth=1.75cm]{geometry}

%% Useful packages
\usepackage{amsmath}
\usepackage{graphicx}
\usepackage{qtree}
\usepackage[colorinlistoftodos]{todonotes}
\usepackage[colorlinks=true, allcolors=blue]{hyperref}
\usepackage{tikz}
\usetikzlibrary{automata,positioning}
\usepackage{amsfonts}
\usepackage{amssymb}
\newcommand\floor[1]{\lfloor#1\rfloor}
\newcommand\ceil[1]{\lceil#1\rceil}

\usepackage{forest}
\renewcommand{\rmdefault}{ptm}

\begin{document}
\section*{Bloom filters vs. space-efficient perfect hash}
Recall that classic Bloom filters use roughly $1.44 log_2(1/f)$ bits per key, as seen in class (where $f = (1 − p)^k$ is the failure probability minimized for $p \approx e^{− \frac{kn}{m}= 1/2}$). The problem asks to extend the implementation required in Problem 10 by employing an additional random universal hash function $s : U \rightarrow [m]$ with $m =\ceil{\frac{1}{f}}$, called signature, so that s(x) is also stored (in
place of x, which is discarded). The resulting space-efficient perfect hash table T has now a one-side error with failure probability of roughly $f$, as in Bloom filters: say why. Design a space-efficient efficient implementation of T, and compare the number of bits per key required by T with that required by Bloom filters.
\\
\\
\textbf{SOLUTION}
\\
\\
Now the data structure has got one-side error with probability because if we use the hash table s, which has got $m$ elements, the probability of a collision is $\frac{1}{m}$. Since, $m=\frac{1}{f}$ we have: $Pr[error]=\frac{1}{\frac{1}{f}}=f$.
\\
\\
We consider first of all to have the same data structure of EX10. Therefore, we have $H,T,A,B,P$, and then till now if we have $n$ keys we occupy a space equal to $18n + o(n)$ bits. Now we add another data structure T' where we store the hash generated by s. T' has got $n$ slots, as the number of keys, in which every element occupy $log_2(m)=log_2(1/f)$ bits, since $s : U \rightarrow [m]$ with $m =\ceil{\frac{1}{f}}$. Now, since this data structure is static, ones we build the $H,T,A,B,P$ we can insert each $s(k)$ in T' in the same position as it is in T. For example, the first element of bucket 1 (if it exist) is going to be the fist element of T', and so on. Then now to find the position of the hash in T' given a key $k$ we use $rank_1(\textsc{base}+\textsc{offset})$, where $\textsc{base}+\textsc{offset}$ is the position calculate to check the if there is a 1 in T ($T[\textsc{base}+\textsc{offset}]$, look EX10).
\\ 
\\
Now we need to check how many bits are used for each key. T' uses $log_2(1/f)$ bit per key as explained before, and, by the previous analysis, we have 18 bit for each key + o(1)( $18n + o(n)$ ). Therefore, we have roughly $log_2(1/f) + 18$ per key, instead Bloom filters use roughly $1.44 log_2(1/f)$. Let's compare them, to see which is more convenient:
\begin{align*}
log_2(1/f) + 18 < 1.44 log_2(1/f) \\
-log_2(f) + 18 < - 1.44 log_2(f) \\
-log_2(f) + 1.44 log_2(f) < - 18) \\
log_2(f) < -\frac{18}{0.44}) \\
f < \frac{1}{2^{\frac{18}{0.44}}} \\
\end{align*}
Then perfect hash is convenient for $f$ less than $\approx 4.84 \times 10^{-13}$.

\end{document}