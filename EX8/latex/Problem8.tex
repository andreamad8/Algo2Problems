\documentclass[a4paper]{article}

%% Language and font encodings
\usepackage[english]{babel}
\usepackage[utf8x]{inputenc}
\usepackage[T1]{fontenc}
\usepackage{algpseudocode}
\usepackage{float}
\usepackage{booktabs}
\usepackage{amsmath}
\newtheorem{theorem}{Theorem}
%% Sets page size and margins
\usepackage[a4paper,top=3cm,bottom=2cm,left=3cm,right=3cm,marginparwidth=1.75cm]{geometry}

%% Useful packages
\usepackage{amsmath}
\usepackage{graphicx}
\usepackage{qtree}
\usepackage[colorinlistoftodos]{todonotes}
\usepackage[colorlinks=true, allcolors=blue]{hyperref}
\usepackage{tikz}
\usetikzlibrary{automata,positioning}
\usepackage{amsfonts}
\usepackage{amssymb}

\usepackage{forest}
\renewcommand{\rmdefault}{ptm}

\begin{document}
\section*{Count-min sketch: extension to negative counters}
Check the analysis seen in class, and discuss how to allow $F[i]$ to change by arbirary values read in the stream. Namely, the stream is a sequence of pairs of elements, where the first element indicates the item $i$ whose counter is to be changed, and the second element is the amount $v$ of that change ($v$ can vary in each pair). In this way, the operation on the counter becomes $F[i] = F[i]+v$, where the increment and decrement can be now seen as $(i, 1)$ and $(i, −1)$.
\end{document}