\documentclass[a4paper]{article}

%% Language and font encodings
\usepackage[english]{babel}
\usepackage[utf8x]{inputenc}
\usepackage[T1]{fontenc}
\usepackage{algpseudocode}
\usepackage{float}
\usepackage{booktabs}
\usepackage{amsmath}
\newtheorem{theorem}{Theorem}
%% Sets page size and margins
\usepackage[a4paper,top=3cm,bottom=2cm,left=3cm,right=3cm,marginparwidth=1.75cm]{geometry}

%% Useful packages
\usepackage{amsmath}
\usepackage{graphicx}
\usepackage{qtree}
\usepackage[colorinlistoftodos]{todonotes}
\usepackage[colorlinks=true, allcolors=blue]{hyperref}
\usepackage{tikz}
\usetikzlibrary{automata,positioning}
\usepackage{amsfonts}
\usepackage{amssymb}

\usepackage{forest}
\renewcommand{\rmdefault}{ptm}

\begin{document}
\section*{Count-min sketch: extension to negative counters}
Check the analysis seen in class, and discuss how to allow $F[i]$ to change by arbitrary values read in the stream. Namely, the stream is a sequence of pairs of elements, where the first element indicates the item $i$ whose counter is to be changed, and the second element is the amount $v$ of that change ($v$ can vary in each pair). In this way, the operation on the counter becomes $F[i] = F[i]+v$, where the increment and decrement can be now seen as $(i, 1)$ and $(i, −1)$.
\\
\\
\textbf{SOLUTION}
\\
\\
The frequency of the item $i$ is represented by $F[i]\rightarrow T[j,h_j(i)]= F[i] + X_{ji}$, where $X_{ji}$ represent the garbage introduce by the other counter. If we just increment the counter the latter quantity is going to be positive, and then we can take the $min_j T[j,h_j(i)]$ to approximate $F[i]$.
Instead, if we have also decrement, it could happen that $X_{ij}<0$ therefore the method for the min is not going to work. In this case, we consider the absolute value of $X_{ji}$(i.e. $|X_{ji}|$) and to approximate $F[i]$ we use $median_j T[j, h_j(i)]$. Now let's proof that, with probability $1-\delta^{1/4}$ holds:
\begin{equation}
F[i] - 3\epsilon \|F\|\leq \hat{F}[i] \leq F[i]  + 3\epsilon \|F\|
\nonumber
\end{equation}
First, let's do some consideration. The value of $\hat{F}[i]=median_j T[j, h_j(i)]$, and $T[j, h_j(i)]= F[i] +|X_{ji}| $. The first inequality(i.e. $F[i] \leq \hat{F}[i]$) holds because we took the absolute value of $X_{ji}$. Now we shall prove that $Pr[\hat{F}[i]> F[i]  + 3\epsilon \|F\|]$. Taken $j$ such that $j=median_j T[j, h_j(i)]$, than we have:
\begin{align*}
Pr&[\hat{F}[i] \leq F[i]  + 3\epsilon \|F\|]\\
Pr&[F[i] - 3\epsilon \|F\|\leq F[i] +|X_{ji}| \leq F[i]  + 3\epsilon \|F\|]\\
Pr&[|X_{ji}|\leq 3\epsilon \|F\|]\\
\end{align*}
From what we have seen in class and by the property of the absolute value, we have $E[|X_{ji}|]\leq E[X_{ji}]= \frac{\epsilon}{e} \|F\|$. Then we can apply the Markov inequality and since universal hash are pairwise independence we have:
\begin{align*}
 Pr[|X_{ji}|>   3\epsilon \|F\| ]< \frac{E[|X_{ji}|]}{3\epsilon \|F\|}< \frac{\frac{\epsilon}{e} \|F\|}{3\epsilon \|F\|}= \frac{1}{3e } < \frac{1}{8}\\
\end{align*}
Let's now define the condition variable $Y=\sum_{j=0}^r Y_j$ which tell us the number of element $i$ (column) that have a garbage $|X_{ji}|$ greater that $\epsilon \|F\|$. 
\begin{equation}
Y_{j} =  
\begin{cases} 
1 \qquad \textsc{if } |X_{ji}|>   3\epsilon \|F\|\textsc{ with } p<\frac{1}{8}\\
0 \qquad \textsc{otherwise}
\end{cases}
\nonumber
\end{equation}
The median of $\hat{F}[i]$ is going to be a good approximation if we don't have more that $\frac{r}{2}$ rows such that $|X_{ji}|>   3\epsilon \|F\|$ ( that is we want $Y < \frac{r}{2}$). \\
Therefore, to calculate the probability of error, we calculate the probability that $Pr[ Y \geq \frac{t}{2}]$. Here, we can use the Chernoff’s Bound
With: $(1+\lambda )\mu =\frac{t}{2}$, $\mu = E[Y]=rp$.
\begin{align*}
\Pr[Y\geq (1+\lambda )\mu] &< \left[{\frac {e^{\lambda }}{(1+\lambda )^{1+\lambda }}}\right]^{\mu}=\left[{\frac{e}{e} \frac {e^{\lambda }}{(1+\lambda )^{1+\lambda }}}\right]^{\mu}=\frac{1}{e^{\mu}}\left[{\frac {e}{(1+\lambda )}}\right]^{(1+\lambda)\mu}=\frac{1}{e^{rp}}\left[{\frac {1}{(1+\lambda )}e}\right]^{\frac{r}{2}}=\frac{1}{e^{rp}}\left[{2pe}\right]^{\frac{r}{2}}\\
\end{align*}
Now we need to prove that $\frac{1}{e^{rp}}\left[{2pe}\right]^{\frac{r}{2}} \leq \delta^{\frac{1}{4}}=\frac{1}{2^{\frac{r}{4}}}$. If we use the reciprocal we have:
\begin{align*}
2^{\frac{r}{4}} \leq \frac{e^{rp}}{\left[{2pe}\right]^{\frac{r}{2}} }\leq \frac{1}{\left[{2pe}\right]^{\frac{r}{2}} }\\
2^{\frac{1}{4}} \leq \frac{1}{ \sqrt{2pe} } 
\end{align*} 
Since $e^{rp}\geq 1$. Then, if we take $\frac{1}{ 2pe } > \sqrt{2} $, that is possible just if $p < \frac{1}{ 2 \sqrt{2}e }$, indeed $p=\frac{1}{8}$.
\end{document}