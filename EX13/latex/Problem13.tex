\documentclass[a4paper]{article}

%% Language and font encodings
\usepackage[english]{babel}
\usepackage[utf8x]{inputenc}
\usepackage[T1]{fontenc}


%% Sets page size and margins
\usepackage[a4paper,top=3cm,bottom=2cm,left=3cm,right=3cm,marginparwidth=1.75cm]{geometry}
\usepackage{amsmath}
\usepackage{amsthm}

\newtheorem{theorem}{Theorem}
\newtheorem{lemma}[theorem]{Lemma}

\begin{document}
\section*{Randomized min-cut algorithm}
Consider the randomized min-cut algorithm discussed in class.
We have seen that its probability of success is at least $1 / \binom{n}{2}$, where $n$ is the number of its vertices.
\begin{itemize}
\item Describe how to implement the algorithm when the graph is represented by adjacency lists, and analyze its running time.
In particular, a contraction step can be done in $O(n)$ time.
\item A weighted graph has a weight $w(e)$ on each edge $e$, which  is  a  positive  real number.
The  min-cut  in  this  case  is  meant  to  be  min-weighted  cut,  where  the sum  of  the  weights  in  the  cut  edges  is  minimum.
Describe  how  to  extend  the algorithm  to  weighted  graphs,  and  show  that  the  probability  of  success  is  still $\geq 1/\binom{n}{2}$. [hint: define the weighted degree of a node]
\item Show that running the algorithm multiple times independently at random, and taking the minimum among the min-cuts thus produced, the probability of success can be made at least $1 - 1/n^c$ or a constant $c > 0$ (hence, with high probability).
\end{itemize}
\
\\
\\
\textbf{SOLUTION}
\\
\\

\noindent
\textbf{First Point}
\\

\noindent
\textbf{Second Point}
\\

\noindent
\textbf{Third Point}
\\

We know that $P(n) \geq 1 / \binom{n}{2}$ is the probability of choose well for $N$ times.
We have seen also that the probability of error is $\leq (1 - 1/\binom{n}{2})$.
Then the probability of having an error each time in $N$ repetitions of the algorithm is 
\begin{align*}
leq (1 - \frac{1}{\binom{n}{2}})^N \approx e^{-\frac{2N}{n(n-1)}}
\end{align*}

Given a $c$ we want the probability of success to be $\geq 1 - 1/n^c$.
This is equivalent to ask that the probability of error is $\leq 1/n^c$.
Making some calculations:
\begin{align*}
\frac{1}{e^{\frac{2N}{n(n-1)}}} = e^{-\frac{2N}{n(n-1)}} &\leq \frac{1}{n^c}\\
n^c &\leq e^{\frac{2N}{n(n-1)}}\\
ln(n^c) &\leq ln(e^{\frac{2N}{n(n-1)}}) = \frac{2N}{n(n-1)}\\
N \geq \frac{n(n-1)ln(n^c)}{2}
\end{align*}
So, for each given $c$, to make the probability of success at least $1 - 1/n^c$ is sufficient to take $N$ grater or equal than $\frac{1}{2} n(n-1)ln(n^c)$.

\end{document}