\documentclass[a4paper]{article}

%% Language and font encodings
\usepackage[english]{babel}
\usepackage[utf8x]{inputenc}
\usepackage[T1]{fontenc}
\usepackage{algpseudocode}
\usepackage{float}
\usepackage{booktabs}
\usepackage{amsmath}
\newtheorem{theorem}{Theorem}
%% Sets page size and margins
\usepackage[a4paper,top=3cm,bottom=2cm,left=3cm,right=3cm,marginparwidth=1.75cm]{geometry}

%% Useful packages
\usepackage{amsmath}
\usepackage{graphicx}
\usepackage{qtree}
\usepackage[colorinlistoftodos]{todonotes}
\usepackage[colorlinks=true, allcolors=blue]{hyperref}
\usepackage{tikz}
\usetikzlibrary{automata,positioning}
\usepackage{amsfonts}
\usepackage{amssymb}
\newcommand\floor[1]{\lfloor#1\rfloor}
\newcommand\ceil[1]{\lceil#1\rceil}

\usepackage{forest}
\renewcommand{\rmdefault}{ptm}

\begin{document}
\section*{Randomized min-cut algorithm}
Consider the randomized min-cut algorithm discussed
in class. We have seen that its probability of success is at least $\frac{1}{\binom{n}{2}}$, where $n$ is the number of its vertices.
\begin{itemize}
\item Describe how to implement the algorithm when the graph is represented by adjacency
lists, and analyze its running time. In particular, a contraction step can
be done in O(n) time.
\item A weighted graph has a weight w(e) on each edge e, which is a positive real
number. The min-cut in this case is meant to be min-weighted cut, where the
sum of the weights in the cut edges is minimum. Describe how to extend the $\frac{1}{\binom{n}{2}}$[hint: define the weighted degree of a node]
\item Show that running the algorithm multiple times independently at random, and
taking the minimum among the min-cuts thus produced, the probability of success
can be made at least 1 − 1/nc
for a constant c > 0 (hence, with high probability).
\end{itemize}
\qquad
\\
\\
\textbf{SOLUTION}
\\
\\

\end{document}