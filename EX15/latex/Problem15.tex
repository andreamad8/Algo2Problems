\documentclass[a4paper]{article}

%% Language and font encodings
\usepackage[english]{babel}
\usepackage[utf8x]{inputenc}
\usepackage[T1]{fontenc}


%% Sets page size and margins
\usepackage[a4paper,top=3cm,bottom=2cm,left=3cm,right=3cm,marginparwidth=1.75cm]{geometry}
\usepackage{amsmath}
\usepackage{amsthm}

\newtheorem{theorem}{Theorem}
\newtheorem{lemma}[theorem]{Lemma}

\begin{document}
\section*{Implicit navigation in vEB layout}
Consider $N = 2h − 1$ keys where $h$ is a power of 2,
and the implicit cache-oblivious vEB layout of their corresponding complete binary
tree, where the keys are suitably permuted and stored in an array of length N without
using pointers (as it happens in the classical implicit binary heap but the rule here
is different). The root is in the first position of the array. Find a rule that, given the
position of the current node, it is possible to locate in the array the positions of its
left and right children. Discuss how to apply this layout to obtain (a) a static binary
search tree and (b) a heap data structure, discussing the cache complexity
\\
\\
\textbf{SOLUTION}
\\
\\


\end{document}
