\documentclass[a4paper]{article}

%% Language and font encodings
\usepackage[english]{babel}
\usepackage[utf8x]{inputenc}
\usepackage[T1]{fontenc}
\usepackage{algpseudocode} 
\usepackage{float}
\usepackage{booktabs}
\usepackage{amsmath}
\newtheorem{theorem}{Theorem}
%% Sets page size and margins
\usepackage[a4paper,top=3cm,bottom=2cm,left=3cm,right=3cm,marginparwidth=1.75cm]{geometry}

%% Useful packages
\usepackage{amsmath}
\usepackage{graphicx}
\usepackage{qtree}
\usepackage[colorinlistoftodos]{todonotes}
\usepackage[colorlinks=true, allcolors=blue]{hyperref}
\usepackage{tikz}
\usetikzlibrary{automata,positioning}

\usepackage{forest}
\renewcommand{\rmdefault}{ptm}

\begin{document}
\section*{Karp-Rabin fingerprinting on strings}
Given a string $S = S[0 \dots n − 1]$, and two positions $0 \leq i < j \leq n − 1$, the longest common extension $lce_S(i, j)$ is the length of the maximal run of matching characters from those positions, namely: if $S[i] != S[j]$ then $lce_S(i, j) = 0$; otherwise, $lce_S(i, j) =  max \{ \textit{l}\geq 1 : S[i \dots i + \textit{l} − 1] = S[j \dots j + \textit{l} − 1] \}$. For example, if $S = abracadabra$, then $lce_S(1, 2) = 0$, $lce_S(0, 3) = 1$, and $lce_S(0, 7) = 4$. Given $S$ in advance for preprocessing, build a data structure for $S$ based on the Karp-Rabin fingerprinting, in O(n log n) time, so that it supports subsequent online queries of the following two types:
\begin{itemize}
\item $lce_S(i, j)$: it computes the longest common extension at positions $i$ and $j$ in O(log n) time.
\item $equalS(i, j,\textit{l})$: it checks if $S[i \dots i + \textit{l} − 1] = S[j \dots j + \textit{l} − 1]$ in constant time.
\end{itemize}
Analyze the cost and the error probability. The space occupied by the data structure can be O(n log n) but it is possible to use O(n) space. [Note: in this exercise, a onetime preprocessing is performed, and then many online queries are to be answered on the fly.]\\
\\
\textbf{SOLUTION}
\\
Consider a string $S$ of length $n$, we call $S[i]$ the i-th element in $S$ starting from $0$, and $S[i,l]$ the substring of $S$ with length $l$ starting at $i$ (i.e. $S[i] \cdot S[i+1] \dots S[i+l-1]$, where $\cdot$ is the string concatenation). The idea is to build an array $R$ with the same length of the string $S$ such that for each index $i$ we have that $R[i]$ contains the prefix of $S$ of length $i + 1$. $R[0]$ contains the first word of $S$, $R[1]$ contains the concatenation of the first two wolds of $S$ etc. In general we will have that $R[i] = S[0,i+1]$. This representation has of course the problem to be too much expensive, but we will see later how to really implement the data structure, first have a look at how it works for checking the equality of two substring and for finding the $lce_S$ of two indexes.
\\
\\
Given a pair of indexes $i$ and $j$, and a length $l$ we want to check whether $S[i,l] = S[j,l]$, looking at $R$.
Considering that $\forall i, l . R[i+l] = S[0,i] \cdot S[i+1,l] = R[i] \cdot S[i+1,l]$, we can prove that
\begin{align*}
\forall i, l\ .\ S[i, l] = R[i + l - 1] - R[i-1]
\end{align*}
where we denote by $\alpha - \beta$ the string $\alpha$ without the prefix $\beta$ (i.e. $\alpha\gamma - \alpha = \gamma$). Now we can simply check whether $s[i,l] = s[j,l]$ by comparing $R[i + l - 1] - R[i-1]$ and $R[j + l - 1] - R[j-1]$.
The problem of finding the longest common extension is quite simple, it is sufficient to use some form of binary search, exploiting the array $R$.
We use a recursive function in order to find the $lce_S(i, j)$, the input are $i$, $j$, the string $S$ and its length $n$.
We consider to have a function $is\_equal(A, x, y, l)$ such that return true if $A[x, l] = A[y, l]$.
The main difference between this algorithm and a binary search is that we will continue calling recursively the function with a string of halved length regardless of the result of the equality check.
\begin{algorithmic}
\Function{lce}{$S$,$i$,$j$,$n$}
\If {$n = 1$}
	\If {\Call {$is\_equal$} {$S$, $i$, $j$, $l$}} 
		\State \Return $1$ 
	\Else 
		\State \Return $0$
	\EndIf
\EndIf
\State $l = n/2$
\If {\Call {$is\_equal$} {$S$, $i$, $j$, $l$}}
	\State \Return $l\ + $ \Call {lce} {$S$,$i + l$,$j + l$,$n - l$}
\Else
	\State \Return \Call {lce} {$S$,$i$,$j$,$l$}
\EndIf
\EndFunction
\end{algorithmic}
Note that assuming the $is\_equal$ function to have constant cost we have that the cost of $lce$ is $O(log(n))$ since at each iteration only a recursive call is reached and the length of the input is halved each time.
\\
\\
In the real implementation, we will use an array $H$ which is the hash fingerprint of $R$.
For each index $i$ $H[i] = h(R[i]) = R[i]\ mod\ p$, of course we don't really built the array $R$, then we can simply fill $H[i]$ with $S[0,i+1]\ mod\ p$.
An implicit assumption in this step (and in Karp-Rabin fingerprint in fact) is to see each string as a number, in particular we see $S$ as the representation of an integer number with base $b$ (depending on the number of possible word in the alphabet).
In the following we will call $S$ the string and $(S)_b$ the number for which $S$ is the representation in base b.
We will also distinguish between $S[i]$ the word in position $i$ of $S$ and $S[i]_b$ the number between $0$ and $b-1$ which represents.
Note that we have to redefine $h$ as $h(\alpha) = (\alpha)_b\ mod\ p$.\\
Then the number $(S)_b$ will be $S[0]_b \times b^{n - 1} + S[1]_b \times b^{n - 2} \dots + S[n-1]_b \times b^{0}$.
With this representation we have the beautiful feature that $(S[0, i+l])_b = (S[0,i] \cdot S[i,l])_b = (S[0,i])_b \times b^l + (S[i,l])_b$, that we can exploit to prove $h(S[i, l]) = H[i + l - 1] - H[i - 1] \times b^l \ mod\ p$.
\begin{align*}				
	     H[i+l-1] =\ & h(R[i+l-1]) \\
	 				& h(S[0, i+l]) \\
	 				& (S[0, i+l])_b\ mod\ p \\
        				& (S[0,i])_b \times b^l + (S[i,l])_b\ mod\ p \\
        				& ((S[0,i])_b\ mod\ p) \times b^l + ((S[i,l])_b\ mod\ p)\ mod\ p \\
        				& h(S[0,i]) \times b^l + h(S[i,l])\ mod\ p \\
        				& h(R[i-1]) \times b^l + h(S[i,l])\ mod\ p \\
        				& H[i-1] \times b^l + h(S[i,l])\ mod\ p \\
\end{align*}
This way the cost of compare two arbitrary substring of the same length $S[i,l]$ and $S[j,l]$ is the cost of acceding to four elements of $H$ plus a constant number of arithmetic operations, i.e. $O(1)$.
Since we have found a good approximation of the procedure $is\_equal$, we can compute $lce_S$ with the algorithm shown before in $O(log\ n)$ time.
\\
The space occupied by the data structure is quasi $O(n)$ where $n$ is the length of $S$. Actually the size of the input in bit is $n$ times the size of the word (i.e. $n \times log\ b$) and the size of the output is $n$ times the size of an element of $H$ (i.e. $n \times log\ p$). Note that we can consider the use of $R$ as a particular case in which $p$ is equal to $b^{n + 1}$ (the maximum number that a string of length n with alphabet of size b can represent plus one), in this case the modulus in $h$ is redundant and we have a data structure of size $O(n^2)$.\\
%Note that the trivial way to fill $H$ takes $O(n^2)$.
%I am referring to something like
%\begin{algorithmic}
%\For{$i \in \{0, 1, \dots n-1\}$}
%	\State $H[i] = 0$
%	\For{$j \in \{0, 1, \dots i\}$}  
%		\State $H[i] = H[i] + S[j]\times b^{i-j}\ mod\ p$
%	\EndFor
%\EndFor
%\end{algorithmic}
%in which we simply implement $H[i] = S[0] \times b^i + S[1] \times b^{i - 1} + \dots S[i] \times b^0$.
%In effect the number of operations (products, sums), is given by  $\sum_{i=1}^n i = \frac{n \times (n+1)}{2} = O(n^2)$.
It is not difficult to see that we can use the compositional property of the $h$ function in order to achieve a cost of $O(n)$ to fill $H$.
The code speaks for itself probably, the idea is to compute $H[0]$ as $S[0]\ mod\ p$, $H[1]$ as $S[1] + S[0] \times b\ mod\ p$ etc.
We use only a constant number of operations for each word in $S$.
\begin{algorithmic}
\State $H[0] = S[0] mod p$
\For{$i \in \{1, \dots n-1\}$}
	\State $H[i] = H[i-1] \times b + S[i]\ mod\ p$
\EndFor
\end{algorithmic}
We take $p$ as a random prime number $p\in [2,\cdots,\tau]$, where $\tau > n$.
Let's have a look at the collision cases: we have a collision when $S[i,l] \neq S[j,l]$ but $h(S[i,l]) = h(S[j,l])$.
This is the same as 
\begin{align*}				
&H[i + l - 1] - H[i - 1] \times b^l \ mod\ p = H[j + l - 1] - H[j - 1] \times b^l \ mod\ p \\
&H[i + l - 1] - H[i - 1] \times b^l - (H[j + l - 1] - H[j - 1] \times b^l) \ mod\ p = 0 \\
&h(S[i,l]) - h(S[j,l])\ mod\ p = 0 \\
&((S[i,l])_b\ mod\ p) - (S[j,l])_b\ mod\ p)\ mod\ p = 0 \\
&(S[i,l])_b - (S[j,l])_b\ mod\ p = 0 \\
&p\ divides\ (S[i,l])_b - (S[j,l])_b \\
\end{align*}
We define $c = (S[i,l])_b - (S[j,l])_b$, we want to count how many bad choices we have for $p$ (how many choices for $p$ such that $p$ divides $c$). 
Note that as $(S)_b$ also $c$ is a number representable with a string of length $n$ with alphabet of size $b$; then $0 \leq c \leq b^n$.
Let $k$ be the number of prime number dividing $c$, then $c = p_1^{i_1} \times p_2^{i_2} \times \dots \times p_k^{i_k}$ and, since $p_x$ is grater or equal than $2$ and $i_x$ is greater or equal than $1$ for all $x$, we can say that $c \geq 2^k$ (as seen in class).
Finally $2^k \leq c \leq b^n \leq b^{n \times log\ b}$, and $k \leq n \times log\ b$.Since the possible choices for a prime number in the interval $[2,\cdots,\tau]$ are approximately $\frac{\tau}{ln(\tau)}$, the probability of error is less or equal to $\frac{\textsc{\#bad primes}}{\textsc{\#primes}}= \frac{n \times log\ b}{\frac{\tau}{ln(\tau)}}$. If we choose $\tau \approx n^{a+1} ln(n)$ then we have that the probability of error is less or equal than $\frac{log\ b}{n^a}$.

%and the size of $H$ is approximatively $n \times log\ p \leq n \times log(n^{a+1} \times ln(n)) = n \times (a+1) \times log(n \times ln(n)) = O(n \times log(n \times log(n)))$ (does it make sense?). Note that if we want the size of $H$ to be $O(n)$ it is necessary for $\tau$ to doesn't depend on $n$, and in this particular case the probability of error grows up with $n$.

\iffalse
Karp-Rabin hashing on strings (i.e. $str[n]$)for solving the longest common extension problem. We have the following steps:
\begin{enumerate}
\item Create a data structure holding the hashes, with hashes of previous characters being held as a prefix. We fix a sufficient big prime $p$ and we use the Karp-Rabin hash (i.e. for a string $k$ and a base $b$: $h(k) = (k[0]b^{L−1} + k[1]b^{L−2}+ \dot+ k [L − 1]b^0) \ mod\  p $) :
\begin{align*}
        H[0]&   = h(str[0]) \\
        H[1]&   = H[0]  p + h(str[1])\\
        H[2]&   = H[1]  p + h(str[2]) \ = \ h(str[0]) p^2 + h(str[1]) p^1 +h(str[2])p^0\\
        &\cdots  \\
        H[n-1]& = H[n-2]  p + h(str[n-1]) \ = \  h(str[0]) p^{n-1} +  \dots  +h(str[n-2]) p^1 + h(str[n-1])p^0\\ 
\end{align*}
Therefore the space used here is just O(n).
\item Firstly we care about equality. To compare equality of a substring of length $l$ at indexes $i, j$, we need to know the sub-hash of the string. Thus, we take hashes of $H[i]$ and $H[i+l]$ and we calculate the hash of the sub-string between $i$ and $l$:
\begin{align*}
H[Substring_i] = &H[i + l] - H[i] * p ^{(l - 1)}\\
=&h(str[0]) p^{i + l-1} +  \dots  +h(str[i + l-2]) p^1 + h(str[i + l-1])p^0\\
-& (h(str[0]) p^{i-1} +  \dots  +h(str[i-2]) p^1 + h(str[i-1])p^0)* p ^{(l - 1)}\\
=& h(str[i + l-1])p^0+h(str[i + l-2]) p^1\dots +h(str[0]) p^{i + l-1}  \\
-& (h(str[i-1])p^{(l - 1)} +h(str[i-2]) p^{l}\dots +h(str[0]) p^{i+l - 2}    )\\
=& h(str[i + l-1])p^0+h(str[i + l-2]) p^1\dots +h(str[i]) p^{i + l-1}  \\   
\end{align*}
We do the same procedure for the sub-string between $j$ and $l$ (i.e. $H[Substring_j]$), the we can simply compare the two hash. We introduce a base case (a sanity check) in the case $l=1$ and we have $str[i] \neq str[j]$. The cost of this operation i O(1) since we did just simple operation ($+ \ *$).\\
We choose a random prime number $p\in [2,\cdots,\tau]$ where $\tau > n$. Since the prime number in the interval $[2,\cdots,\tau]$ are approximately $ \frac{\tau}{ln(\tau)}$, and we have a collision when $Substring_i=Substring_j$ but $H[Substring_i]\neq H[Substring_j]$, thus when $c= Substring_i- Substring_j \ mod \ p=0$. We can conclude that $P_r[error] \leq \frac{\textsc{\#bad prime}}{\textsc{\#prime}}= \frac{n}{\frac{\tau}{ln(\tau)}}$ because there are at most $n$ distinct prime $p$ that divide $c$ (Chinese Theorem of residual). If we choose $\tau \approx n^{a+1} ln(n)$ then we have $P_r[error] \leq \frac{1}{n^a}$.
\item Finally to compute the the longest common extension $lce_S(i, j)$, we just do a binary search of the index $l$. The cost to do so is O(ln(n)) since the check of the equality is constant.
\end{enumerate}
\fi
\newpage \qquad \\
\\
\textbf{SOLUTION in O(N LOG N) works for every kind of hash function}
\\
The proposed data structure that maintain a series of trees. At each level we encode power of two elements (THIS IS IMPOSSIBLE TO EXPLAIN... LOOK THE CODE). The space occupied by this data structure is $O(nlog(n))$ where n is the length of the input array.   
\begin{algorithmic}
\Function{CreateTree}{$S$,$n$,$p$}
\State $A \gets NEW \ Array[log_2(n)+1]$
\State $TEMP \gets NEW \ Array[n]$
\For{$i \ IN \ (0,n)$}  
\State $TEMP[i] \gets S[i] \ mod \ p$
\EndFor 
\State $A[0] \gets TEMP$
\For{$i \ IN \ (1,log_2(n))$} 
\State $TEMP \gets NEW \ Array[n-i]$ 
\For{$j \ IN \ (0,n-i)$}  
\State $TEMP[j] \gets A[i-1][j] + A[i-1][j+2^{i-1}] \ mod \ p$
\EndFor 
\State $A[i] \gets TEMP$
\EndFor 
\State \Return $A$
\EndFunction
\end{algorithmic}
Now to have $lce_S(i, j)$ we build the implement the following procedure, where $h=\lfloor log_2(n-j)\rfloor$
\begin{algorithmic}
\Function{LCE}{$i$,$j$,$h$}
\If {$A[h][i]== A[h][j]$}
    \State \Return $2^h$
\Else
    \If {$h \neq 0$}
        \State \Return $0$
    \Else
    	\State \Return LCE$(i, j,h-1) +$LCE$(i+2^{h-1},j+2^{h-1},h-1)$ 
    \EndIf
\EndIf
\EndFunction
\end{algorithmic}
Notice that, this procedure work in $O(log(n))$ since the array is length is at most log(n) and we are doing two recursive call with an array one unit smaller each time. Finally to obtain $equalS(i, j,\textit{l})$ in cost $O(1)$ we simply check whether $A[\lceil log_2(l) \rceil )][i]== A[\lceil log_2(l) \rceil ][j]$ is true. \\
Notice that all this algorithm work for arrays in which their length $n$ is a power of two. If we have an array that is not of the latter length, we are doing the following: create the same data structure as before but the part of the array that is not in the tree, that it's at most long $2^{i+1}-2^{i}-1$ where $i= \lfloor log_2(n)\rfloor$, is store in simple array. In this case whether we need to check if $A[\lfloor log_2(n-j)\rfloor][i]== A[\lfloor log_2(n-j)\rfloor][j]$, i.e. the longest possible string, we need also to check, manually, whether there is a matching in the array. Last but not least, the calculation of the error probability is exactly the same to the one analysed during the course.  
\end{document}