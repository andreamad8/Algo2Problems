\documentclass[a4paper]{article}

%% Language and font encodings
\usepackage[english]{babel}
\usepackage[utf8x]{inputenc}
\usepackage[T1]{fontenc}

\usepackage{tikz}
\usetikzlibrary{matrix}
\usepackage{algpseudocode} 

%% Sets page size and margins
\usepackage[a4paper,top=3cm,bottom=2cm,left=3cm,right=3cm,marginparwidth=1.75cm]{geometry}
\usepackage{amsmath}
\usepackage{amsthm}
\usepackage{amssymb}

\newtheorem{theorem}{Theorem}
\newtheorem{lemma}[theorem]{Lemma}

\begin{document}
\section*{1-D range query}
Describe how to efficiently perform one-dimensional range queries
for the data structures described in Problems 14 and 15. Given two keys $k_1 \leq k_2$, a
range query asks to report all the keys $k$ such that $k_1 \leq k \leq k_2$. Give an analysis of the
cost of the proposed algorithm, asking yourself whether it is output-sensitive, namely,
it takes $O(log_B N + R/B)$ block transfers where $R$ is the number of reported keys.
\
\\
\\
\textbf{SOLUTION}
\\
\\


\end{document}