\documentclass[a4paper]{article}

%% Language and font encodings
\usepackage[english]{babel}
\usepackage[utf8x]{inputenc}
\usepackage[T1]{fontenc}
\usepackage{algpseudocode} 
\usepackage{float}
\usepackage{booktabs}
\usepackage{amsmath}
\newtheorem{theorem}{Theorem}
%% Sets page size and margins
\usepackage[a4paper,top=3cm,bottom=2cm,left=3cm,right=3cm,marginparwidth=1.75cm]{geometry}

%% Useful packages
\usepackage{amsmath}
\usepackage{graphicx}
\usepackage{qtree}
\usepackage[colorinlistoftodos]{todonotes}
\usepackage[colorlinks=true, allcolors=blue]{hyperref}
\usepackage{tikz}
\usetikzlibrary{automata,positioning}

\usepackage{forest}
\renewcommand{\rmdefault}{ptm}

\begin{document}
\section*{Hashing sets}
Your company has a database $S \in U$ of keys. For this database, it uses a randomly chosen hash function $h$ from a universal family $H$ (as seen in class); it also keeps a bit vector $B_S$ of $m$ entries, initialized to zeroes, which are then set $B_S[h(k)] = 1$ for every $k \in S$ (note that collisions may happen). Unfortunately, the database $S$ has been lost, thus only $B_S$ and $h$ are known, and the rest is no more accessible. Now, given $k \in U$, how can you establish if $k$ was in $S$ or not? What is the probability of error? Under the hypothesis that $m \geq c|S|$ for some $c > 1$ (note: we do not know the actual values of $c$ and $|S|$) can you estimate the size $|S|$, i.e. the size of $S$, looking at just $h$ and $B_S$? What is the probability of error? Note that $S$ is no more accessible as it disappeared.\\ 
Optional: Another database $R$ has been found to be lost: it was using the same hash function $h$, and the bit vector $B_R$ defined analogously as above. Using $h$, $B_S$, and $B_R$, how can you establish if $k$ was in $S \cup R$ (union), $S \cap R$ (intersection), or $S \setminus R$ (difference)? What is the probability of error?\\
\\
\textbf{SOLUTION}
\\

\begin{itemize}
\item[\textbf{a)}] To check whether $k\in U$ belong to $S$, we simply check $B_S[h(k)]=1$. The probability of error is equal to $P(error)=1-(1 - \frac{1}{m})^{|S|}$. This problem is similar to the "Birthday paradox" (i.e. fixed a day how many people are born on the same day). In word: the probability of a collision is $\frac{1}{m}$, thus the probability to do not have a collision is $(1-\frac{1}{m})$. If we have $|S|$ key the probability do not have any collision is $(1 - \frac{1}{m})^{|S|}$. Therefore the probability to have a collision is: $1-(1 - \frac{1}{m})^{|S|}$. That is, the probability that there is at least one collision with of the key is S, the probability that $\exists j\in S : h(k)=h(j)$ but $j\neq k$ with $k \in U$.
\item[\textbf{b)}] To estimate the size of $S$, we first create an indicator variable $X=\sum_{i=0}^{m-1}X_i$ where
\begin{equation}
X_{i} =  
\begin{cases} 
1 \qquad \textsc{if }B_S[h(k)]=1\\
0 \qquad \textsc{otherwise}
\end{cases}
\nonumber
\end{equation}
Than the expectation $E[X]$ represents the expected number of $1$ in the $B_s$ table. To calculate the expectation we need to estimate the $P(B_S[h(k)]=1)$, that, by the point a), should be $(1 - \frac{1}{m})^{|S|}$. Since we do not know $|S|$ we use the hypothesis, that is $|S|\leq \frac{m}{c}$. Therefore we have:
\begin{equation}
P(B_S[h(k)]=1)=(1 - \frac{1}{m})^{|S|} \leq (1 - \frac{1}{m})^{\frac{m}{c}}
\nonumber
\end{equation}
Hence we have $E[X]= \sum_{i=0}^{m-1} (1 - \frac{1}{m})^{\frac{m}{c}}=m(1 - \frac{1}{m})^{\frac{m}{c}}$. Now we have got a bound for $|S|$: $E[X]\leq|S|\leq \frac{m}{c}$.
\item[\textbf{c)}] For the optional point we have:
	\begin{itemize}
	\item[\textbf{Union}] We need to check that $B_S[h(k)]=1 $ OR $B_R[h(k)]=1$ and then the probability of error is $P[B_S[h(k)]=1 \lor B_R[h(k)]=1]$ that , by set theory is equal to $P[B_S[h(k)]=1] +P[B_S[h(k)]=1] - P[B_S[h(k)]=1 \land B_R[h(k)]=1]$
	\item[\textbf{Intersection}] We need to check that $B_S[h(k)]=1 $ AND $B_R[h(k)]=1$ and then the probability of error is $P[B_S[h(k)]=1 \land B_R[h(k)]=1]$ 
	\item[\textbf{Difference}] We need to check that $B_S[h(k)]=1 $ AND $B_R[h(k)]=0$ and then the probability of error is $P[B_S[h(k)]=1 \land B_R[h(k)]=0]$ that , by set theory is equal to  $P[B_S[h(k)]=1] - P[B_S[h(k)]=1 \land B_R[h(k)]=1]$
	\end{itemize}
\end{itemize}
\end{document}
