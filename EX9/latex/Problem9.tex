\documentclass[a4paper]{article}

%% Language and font encodings
\usepackage[english]{babel}
\usepackage[utf8x]{inputenc}
\usepackage[T1]{fontenc}
\usepackage{algpseudocode}
\usepackage{float}
\usepackage{booktabs}
\usepackage{amsmath}
\newtheorem{theorem}{Theorem}
%% Sets page size and margins
\usepackage[a4paper,top=3cm,bottom=2cm,left=3cm,right=3cm,marginparwidth=1.75cm]{geometry}

%% Useful packages
\usepackage{amsmath}
\usepackage{graphicx}
\usepackage{qtree}
\usepackage[colorinlistoftodos]{todonotes}
\usepackage[colorlinks=true, allcolors=blue]{hyperref}
\usepackage{tikz}
\usetikzlibrary{automata,positioning}
\usepackage{amsfonts}
\usepackage{amssymb}

\usepackage{forest}
\renewcommand{\rmdefault}{ptm}

\begin{document}
\section*{Count-min sketch: range queries}
Show and analyse the application of count-min sketch to range queries $(i,j)$ for computing  $\sum^j_{k=i} F[k]$. Hint: reduce the latter query to the estimate of just $t \leq 2\ log\ n$ counters $c_1,c_2,...,c_t$. Note that in order to obtain a probability at most $\delta$ of error (i.e. that $\sum^t_{l=1}c_l > \sum^j_{k=i}F[k] + 2\epsilon\ log\ n ||F||$), it does not suffices to say that it is at most $\delta$ the probability of error of each counter $c_l$: while each counter is still the actual wanted value plus the residual as before, it is better to consider the sum $V$ of these $t$ wanted values and the sum $X$ of these residuals, and apply Markov's inequality to $V$ and $X$ rather than on the individual counters.
\\
\\
\textbf{SOLUTION}
\\
\\
We use the Dyadic Range: a dyadic range is a range of the form $[x2^y+1 . . .(x+1)2^y]$  for parameters x and y. Each point in the range $[1\dots n]$ is a member of $log_2 n$ dyadic ranges, one for each y in the range $0\dots log_2(n) − 1$ (think it as tree, where we slit the interval in 2 at each level).
A count-min sketch table is kept for each set of dyadic ranges of length $2^y$, one for each level in the tree, thus we have $log_2 n$ CM sketches.
\\
\\
Therefore, each time an element arrive we will update all the $log_2 n$ tables. When instead we have a range query $[i,j]$,  we compute the at most $2 log_2 n$ dyadic ranges which canonically cover the range (this is the same of the exercise 1). We can think at them as many point queries to the CM sketch table, that with their sum gives the estimate result. [this is quite hard to explain, have a look to this \href{http://dimacs.rutgers.edu/~graham/pubs/papers/cm-full.pdf}{LINK}]
\\
\\
Therefore for each query we have $t \leq 2\ log\ n$ counters $c_1,c_2,...,c_t$. We then, want to proof $Pr[\sum^t_{l=1}c_l > \sum^j_{k=i}F[k] + 2\epsilon\ log\ n ||F||]<\delta$. Firstly, we notice that each counter $c_i$ represents an interval $[a,b]$ and it value is $\sum_{k=a}^b (F[k]) + X_i$, where $X_i$ is the rubbish of that particular interval. Then, give an interval $[l,r]$, and the counters we have $\sum^t_{i=1}c_i = \sum^r_{k=l}F[k] +X $, where X is the total error (notice that this works because the interval are disjoint). Then we substitute to the previous equation and we have:
\begin{align*}
Pr &[\sum^t_{l=1}c_l > \sum^r_{k=l}F[k] + 2\epsilon\ log\ n ||F||]<\delta \\
Pr &[ \sum^r_{k=l}F[k] +X  > \sum^r_{k=l}F[k] + 2\epsilon\ log\ n ||F||]<\delta \\
Pr &[ X  >  2\epsilon\ log\ n ||F||]<\delta \\
\end{align*}
Now, we apply Markov inequality and by the linearity of the expectation we have: $Pr [ X  >  2\epsilon\ log\ n ||F||] \leq \frac{E[X]}{2\epsilon\ log\ n ||F||} \leq \frac{\sum_{i}^{2log_2(n)} E[X_i]}{2\epsilon\ log\ n ||F||}$. Since, for each $E[X_i]< \frac{\epsilon}{e}||F||$ we have $\frac{\sum_{i}^{2log_2(n)} E[X_i]}{2\epsilon\ log\ n ||F||} \leq \frac{2 \frac{\epsilon}{e}log \ n||F||}{2\epsilon\ log\ n ||F||}= \frac{1}{e}$. This is the probability of error (that the sum of garbage is more the $2\epsilon\ log\ n ||F||$) in row $j$ (the min).
\\
\\
Since we choose the row of each table that minimize the sum of the counter (by definition), then there must be an error in all the row $r$. Thus we have $\frac{1}{e^r}=\delta$ since $r=ln(\frac{1}{\delta})$.

\end{document}