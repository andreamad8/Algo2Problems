\documentclass[a4paper]{article}

%% Language and font encodings
\usepackage[english]{babel}
\usepackage[utf8x]{inputenc}
\usepackage[T1]{fontenc}
\usepackage{algpseudocode}
\usepackage{float}
\usepackage{booktabs}
\usepackage{amsmath}
\newtheorem{theorem}{Theorem}
%% Sets page size and margins
\usepackage[a4paper,top=3cm,bottom=2cm,left=3cm,right=3cm,marginparwidth=1.75cm]{geometry}

%% Useful packages
\usepackage{amsmath}
\usepackage{graphicx}
\usepackage{qtree}
\usepackage[colorinlistoftodos]{todonotes}
\usepackage[colorlinks=true, allcolors=blue]{hyperref}
\usepackage{tikz}
\usetikzlibrary{automata,positioning}
\usepackage{amsfonts}
\usepackage{amssymb}

\usepackage{forest}
\renewcommand{\rmdefault}{ptm}

\begin{document}
\section*{Space-efficient perfect hash}
Consider the two-level perfect hash tables presented in [CLRS] and discussed in class. As already discussed, for a given set of $n$ keys from
the universe $U$, a random universal hash function $h : U \rightarrow [m]$ is employed where $m = n$, thus creating $n$ buckets of size $n_j \geq 0$, where $\sum^{n−1}_{j=0} nj = n$. Each bucket $j$ uses a random universal hash function $h_j: U \rightarrow [m]$ with $m = n_j^2$. Key $x$ is thus stored in
position $h_j (x)$ of the table for bucket $j$, where $j = h(x)$.
This problem asks to replace each such table by a bitvector of length $n=n_j^2$, initialized to all 0s, where key $x$ is discarded and, in its place, a bit 1 is set in position $h_j (x)$ (a similar thing was proposed in Problem 4 and thus we can have a one-side error). Design
a space-efficient implementation of this variation of perfect hash, using a couple of tips. First, it can be convenient to represent the value of the table size in unary (i.e., x zeroes followed by one for size x, so 000001 represents x = 5 and 1 represents x = 0). Second, it can be useful to employ a rank-select data structure that, given any bit vector B of b bits, uses additional o(b) bits to support in O(1) time the following operations on B:
\begin{itemize}
\item $rank_1(i)$: return the number of 1s appearing in the first i bits of B.
\item $select_1(j)$: return the position i of the jth 1, if any, appearing in B (i.e. B[i] = 1 and $rank_1(i) = j$).
\end{itemize}
Operations $rank_0(i)$ and $select_0(j)$ can be defined in the same way as above. Also, note that o(b) stands for any asymptotic cost that is smaller than $\Theta(b)$ for $b \rightarrow \inf$.
\\
\\
\textbf{SOLUTION}
\\
\\

\end{document}