\documentclass[a4paper]{article}

%% Language and font encodings
\usepackage[english]{babel}
\usepackage[utf8x]{inputenc}
\usepackage[T1]{fontenc}
\usepackage{algpseudocode}
\usepackage{float}
\usepackage{booktabs}
\usepackage{amsmath}
\newtheorem{theorem}{Theorem}
%% Sets page size and margins
\usepackage[a4paper,top=3cm,bottom=2cm,left=3cm,right=3cm,marginparwidth=1.75cm]{geometry}

%% Useful packages
\usepackage{amsmath}
\usepackage{graphicx}
\usepackage{qtree}
\usepackage[colorinlistoftodos]{todonotes}
\usepackage[colorlinks=true, allcolors=blue]{hyperref}
\usepackage{tikz}
\usetikzlibrary{automata,positioning}
\usepackage{amsfonts}
\usepackage{amssymb}
\newcommand\floor[1]{\lfloor#1\rfloor}
\newcommand\ceil[1]{\lceil#1\rceil}

\usepackage{forest}
\renewcommand{\rmdefault}{ptm}

\begin{document}
\section*{External memory implicit searching}
Given a static input array A of N keys in the
EMM (external memory or cache-aware model), describe how to organize the keys
inside A by suitably permuting them during a preprocessing step, so that any subsequent
search of a key requires $O(log_B N)$ block transfers using just $O(1)$ memory words
of auxiliary storage (besides those necessary to store A). Clearly, the CPU complexity
should remain $O(log N)$. Discuss the I/O complexity of the above preprocessing,
assuming that it can uses $O(N/B)$ blocks of auxiliary storage. (Note that the additional
$O(N/B)$ blocks are employed only during the preprocessing; after that, they are
discarded as the search is implicit and thus just O(1) words can be employed.)
\\
\\
\textbf{SOLUTION}
\\
\\
\end{document}