\documentclass[a4paper]{article}

%% Language and font encodings
\usepackage[english]{babel}
%\usepackage[utf8x]{inputenc}
%\usepackage[T1]{fontenc}
%\usepackage{algpseudocode}
%\usepackage{float}
%\usepackage{booktabs}
%\usepackage{amsmath}
%\newtheorem{theorem}{Theorem}
%% Sets page size and margins
%\usepackage[a4paper,top=3cm,bottom=2cm,left=3cm,right=3cm,marginparwidth=1.75cm]{geometry}

%% Useful packages
\usepackage{amsmath}
%\usepackage{graphicx}
%\usepackage{qtree}
%\usepackage[colorinlistoftodos]{todonotes}
%\usepackage[colorlinks=true, allcolors=blue]{hyperref}
%\usepackage{tikz}
%\usetikzlibrary{automata,positioning}
%\usepackage{amsfonts}
%\usepackage{amssymb} 

%\usepackage{forest}
%\renewcommand{\rmdefault}{ptm}

\begin{document}
\section*{Special case of most frequent item in a stream}
Suppose to have a stream of $n$ items, so that one of them occurs $> n/2$ times in the stream. Also, the main memory is limited to keeping just two items and their counters, plus the knowledge of the value of $n$ beforehand. Show how to find deterministically the most frequent item in this scenario. [Hint: since the problem cannot be solved deterministically if the most frequent item occurs $\leq n/2$ times, the fact that the frequency is $> n/2$ should be exploited.]
\\
\\
\textbf{SOLUTION}
\\
\\
This solution works keeping in memory only one counter \textbf{$c$} and a stream alphabet value \textbf{$v$}.

\noindent 
Consider the stream as an array $S$ of elements.
The algorihm will work this way for each element in S:
\begin{enumerate}
\item set $i = 1$, $c = 1$, $v = s[0]$
\item if $c = 0$ then set $c = 1$ and change the value $v = S[i]$
\item if $c>0 \land S[i] = v$  we increase the counter $c$ 
\item if $c>0 \land S[i] \neq v$ we decrease the counter $c$
\item if there is more stream to analyze increase $i$ and repeat from step $2.$ else return $v$ as the most frequent item
\end{enumerate} 


\paragraph{obs1} 
The counter increases each time we find an element $v'$ in the stream such that $v = v'$ and decreases each time we find $v \neq v'$.

\

So when the counter $c = 0$  the algorithm has found a portion of stream where there are as many $v' \neq v$ as those such that $v' = v$.
The portion starts from the position where we changed the value of $v$ and ends where the counter was set to $0$.
Every time the counter goes to $0$ we can define such portion. 

\noindent 
Let's call $P_i$ the $i$-th parition created and $l_i$ its length. Defining $f_i(v_i)$ as the frequency of (the number of occurence of) the element $v_i$ from \textbf{obs1} we can say that 
\begin{equation}
f_i(v_i) = \sum_{x \in P_i \\ i \neq v_i} f_i(x)
\end{equation}
that implies that
\begin{equation}
f_i(x) \leq f_i(v_i)
\end{equation}

\

\noindent
This means that for each element $x \in P_i$ such that $x \neq v_i$
\begin{equation}
f_i(x) \leq  \frac{l_i}{2}%\left \lceil{\frac{l_i}{2}} \right \rceil
\end{equation}
i.e. all the element in a portion will occur less than half the length of that portion

\paragraph{obs2} When the algorithm stops, $c>0$. Infact if the algorithm would stop with $c=0$ it would mean that for all $x \in S$
\begin{displaymath}
\sum_{i=1}^{|P|} f_i(x) \leq \frac{l_i}{2} = \frac{n}{2} 
\end{displaymath}
where $|P|$ is the number of portions created, and that is not compatible with the problem hypothesis that there is one element that appears more than $n/2$.


\paragraph{obs3} 
At the end of the execution we can then identify another portion, say $P_k$, that begins the last time the value $v$ changed and finishes at the last element of the stream.
In this portion the element $v_k$ is such that it occurs \emph{more} than the sum of all other:
\begin{equation}
f_k(v_k) > \sum_{x \in P_k} f_k(x)
\end{equation}
and so the algorithm will split the stream in $k$ portions such that $k-1$ ends with counter set to $0$ and the $k$-th with $c > 0$.

\

\noindent
From all this observation we can now prove that, since the algorithm will stop with $c>0$, the value $v = v_k$ returned from it will be the most frequent element.

\noindent
Infact, since the frequency of an element $x$ will be $f(x) = \sum_{i=1}^k f_i(x)$, we know that if $x \neq v_k$
\begin{align*}
f(x) &= \sum_{i=1}^k f_i(x) \leq \\
&\leq \sum_{i=1}^{k} f_i(v_i) \leq &\texttt{from \textit{(2)}}\\
&\leq \sum_{i=1}^{k} \frac{l_i}{2} = \frac{n}{2}  &\texttt{from \textit{(3)}}
\end{align*}

Since there can be only one element $x^*$ such that $f(x^*) > n/2$ it will necessary be $v_k$ and then the algorithm is correct.
\end{document}
