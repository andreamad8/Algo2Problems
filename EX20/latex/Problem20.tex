\documentclass[a4paper]{article}

%% Language and font encodings
%\usepackage[english]{babel}
%\usepackage[utf8]{inputenc}
%\usepackage{float}
%\usepackage{tikz}
%% Sets page size and margins
\usepackage[a4paper,top=3cm,bottom=2cm,left=3cm,right=3cm,marginparwidth=1.75cm]{geometry}
%\usepackage{amsmath}
%\usepackage{amssymb}

\begin{document}
\section*{Wrong greedy for minimum vertex cove}
Find an example of (family of) graphs for which the following greedy approach fails to give a 2-approximation for the minimum vertex cover problem (and prove why this is so). 
Start out with an empty $\tilde{S}$.
Choose each time a vertex $v$ with the largest number of incident edges in the current graph.
Add $v$ to $\tilde{S}$ and remove its incident edges.
Repeat the process on the resulting graph as long as there are edges in it.
Return $|\tilde{S}|$ as the approximation of the minimal size of a vertex cover for the original input graph.

Generalize your argument to show that the above greedy algorithm cannot actually provide an r-approximation for any given constant $r > 1$.
\\
\\
\textbf{Solution}
\\
\\
We will use a partitioned graph $G = (V, E)$.
The set of vertices $V$ is divided in two partitions $A$ and $B$.
The graph is such that all the edges connect two vertices one from $A$ and one from $B$ (i.e. we don't allow intra-partition edges).

In this way we can see that the degree of $A$ and $B$ are the same (we define the degree of a set of vertices to be the sum of the degrees of the vertices), they are exactly $|E|$.
We name $n = |B|$ and $m = |A|$.
Note that if every node is touched by at least one edge (it will be the case for our graph), then one minimum vertex cover is exactly $A$ or $B$ depending on which is the minimum between $m$ and $n$.

We will build the graph in a way such that the minimum vertex cover is $B$ (i.e. $m > n$), but the greedy algorithm can choose all the nodes of $A$ as minimum vertex cover, and we will show that $m$ can be strictly greater than $2n$ if $n$ is sufficiently large.

\

%
%
%Consider a graph with $k = |E|$ edges, the best way to build $B$ is to equally distribute the edges on its nodes in order to minimize the maximum degree of each node.
%In $A$ instead we need to distribute the edges in a very non-balanced way between the its nodes, in order to have few nodes with .
The idea is to fix $n$ and then build the graph having $m$ depending somehow on $n$, such that the $m$ node of $A$ can be chosen by the algorithm. 

The construction works this way:
\begin{enumerate}
\item set $i = 1$, $A = \{\}$
\item create $\lfloor n/i \rfloor$ new nodes with degree $i$ 
\item insert the nodes obtained in $A$
\item connect each edge of those nodes to a \emph{different} node in $B$ (e.g. with a round robin policy\footnote{this roundrobin policy is to be considered over the whole execution of the algorithm, i.e. at iteration $i$ we will start connecting the new edges to the node in $B$ ``immediatly after'' the lastone connected at the $i-1$ iteration.})
\item increase $i$, if it is not greater than $n$ then go to $1.$ else you are done
\end{enumerate}

Following this construction the following facts are true:

\paragraph{fact 1}At the $i$-th iteration we add a total number of new edges \emph{less or equal} than $n$, each one connected to a different node in $B$.

\paragraph{fatc 2}Since we distribute the edges in a uniform way on the nodes in $B$, the degree of a node in $B$ at iteration $i$ will be less or equal than $i$ (i.e. for each iteration we add zero or one edge to each node in $B$)
\paragraph{fact 3}A node created at iteartion $i$ has exactly degree $i$.

\

\noindent Now let's study what happends when we run the greedy algorithm on the graph obtained in the proposed way.

At the very beginning the graph is the one obtained after the n-th iteration of the construction algorithm. 
The greedy algorithm will select a node between one of the nodes in $B$ (each of them has degree less or equal than $n$ for \textbf{fact2}) or the last node inserted in $A$, which has degree $n$ exactly (for \textbf{fact3}).
If the greedy algorithm select this last node from $A$ the algorithm will remove all the edges touching this node (and the node itself) and so each node in $B$ will have its degree decreased by $1$.
The remaining graph will be the one obtained after the $(n-1)$-th iteration of the construction!

\

By construction we know that at iteration $n-1$ we have $\lfloor n/n-1 \rfloor$ nodes in $A$ each of which with degree $n-1$.
From \textbf{fact2} we also know that at the $(n-1)$-th iteration the maximum degree of the nodes in $B$  is \emph{at most} $n-1$.

We are in the same situation of the very beginning and the greedy algorithm can choose again a node in $A$.
This time we can know that the greedy algorithm can choose from $A$ not only one time, but $\lfloor n/n-1 \rfloor$ times, and choosing all the nodes from $A$ (removing those nodes and edges) will cause the graph to return in the state it was at iteration $n-2$ of the building algorithm.

With the same reasoning of above we can observe that at any iteration of the greedy algorithm the graph will have a node in $A$ which has degree greater or equal than the degree of each node in $B$, hence legal to be chosen\footnote{of course the greedy algorithm could select a note in $B$ as well, but this will not affect our argument because we are trying to prove that there is a sequence of possible wrong choices.}.

\

Therefore the greedy algorithm running on a graph constructed as described above can return $m = |A|$ as approximation of the MVC of that graph, i.e. there exists a sequence of choices that will select all the nodes in $A$. 
If $m > 2n$ the result will not be a $2$-approximation of MVC.
We can see that this happens with $n \geq 6$

Infact we can see that with our construction $A$ will have $m = 6 + \frac{6}{2} + \frac{6}{3} + \lfloor \frac{6}{4} \rfloor +   \lfloor \frac{6}{5} \rfloor + 1 = 14$ nodes and so the greedy algorithm can return $14$ that is not a 2 approximation of the MVC problem.
\\
\\
\textbf{Generalization}
\\
We want to show that for any $r>1$ there exist a graph such that the greedy algorithm will not provide an $r$-approximation running on it.

What we have defined is a sort of class of graph that depends only on the parameter $n$ of nodes in $B$.
We have seen before that for $n = 6$ the graph obtained is not good for a $2$-approximation using the greedy algorithm.

We can generalize the solution showing that for any $r > 1$ it is sufficient to take an $n$ big enough to have a graph for which the greedy algorithm can fail in providing a valid $r$-approximation.

\

Using our construction algorithm, given $n$ (which is the correct size of the MVC), the number of nodes in $A$ will be 
\[
m = \sum_{i=1}^n \lfloor n/i \rfloor
\]

so we want to show that 
\[
\sum_{i=1}^n \lfloor n/i \rfloor > rn
\]


We can observe that\footnote{remember that $\sum_{i=1}^n \frac{1}{n}$ is the nth partial sum of the diverging harmonic series, and that it holds
\[
\sum_{i=1}^n \frac{1}{n} > ln(n + 1)
\]
.}	

\[
\sum_{i=1}^n \lfloor \frac{n}{i} \rfloor > \sum_{i=1}^n (\frac{n}{i} - 1) = n \sum_{i=1}^n \frac{1}{i} - n = n(-1 + \sum_{i=1}^n \frac{1}{i}) > n(\ln{(n + 1)} - 1) \approx n(\ln{(n)})
\]

Since $nlogn$ grows faster than $rn$, for any given $r>1$, there will always be a value of $n$ that cause the greedy algorithm to provide a value $m = \sum_{i=1}^n \lfloor n/i \rfloor > rn$ that is not an $r$-approximation.



\end{document}
