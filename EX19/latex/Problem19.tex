\documentclass[a4paper]{article}

%% Language and font encodings
\usepackage[english]{babel}
\usepackage[utf8x]{inputenc}
\usepackage[T1]{fontenc}
\usepackage{float}
\usepackage{tikz}
\usetikzlibrary{matrix}
\usepackage{algpseudocode} 

%% Sets page size and margins
\usepackage[a4paper,top=3cm,bottom=2cm,left=3cm,right=3cm,marginparwidth=1.75cm]{geometry}
\usepackage{amsmath}
\usepackage{amsthm}
\usepackage{amssymb}

\newtheorem{theorem}{Theorem}
\newtheorem{lemma}[theorem]{Lemma}

\begin{document}
\section*{Suffix sorting in EM}
Using the DC3 algorithm seen in class, and based on a variation
of mergesort, design an EM algorithm to build the suffix array for a text of $N$
symbols. The I/O complexity should be the same as that of standard sorting, namely,
$O(\frac{N}{B} log_{\frac{M}{B}} \frac{N}{B})$ block transfers.
\\
\\
\textbf{Solution}
The main idea for solving this problem is, as Prof. Grossi said in class, to prepermute the datas in external memery in order to load them in main memory in linear time (in EM, i.e. $O(\frac{N}{B})$).

Let follow the steps DC3 algorithm is compesed of, and show in which parts a permutation is necessary(see the third section of \footnote{https://www.cs.helsinki.fi/u/tpkarkka/publications/jacm05-revised.pdf}).

\begin{enumerate}
\item \textbf{Construct a sample: } In this step we need to pair the sample positions with the corrisponding 3-gram, since we cannot access in reasonable time to a random position in following sorts. Now we have for $k = 0,1,2$ the sets of pairs:
$$B_k = \bigl\{(i, s[i]s[i+1]s[i+2]) \bigm| i \; mod \; 3 = k)\bigr\}$$ 
Let $C = B_1 \cup B_2$ be the sample set.
This can be made with linear number of block transfers by scanning the input string $S$.

\item \textbf{Sort sample suffixes: } First sort the suffixes using a mixture of radix and merge sort. That operation takes $O(\frac{N}{B} log_{\frac{M}{B}} \frac{N}{B})$.
Now let generate the ids for the renaming steps by scanning the sorted suffixes (Note that in general $rank(i) \neq id(i)$ with $(s, i) \in C$ since there can be duplicates). 
Then permute the pairs $(id(i), s)$ by the congruence class of $i$ and if equals by $i$, where $(i, s) \in C$.
If there is no duplicates we have a total order for them, otherwise the recursion starts, constructing the input string simply concatenating the just permuted ids.
In both cases at the end of this step we have the correct ranks of suffixes in $C$.

\item \textbf{Sort nonsample suffixes: } 
First permute the ranks obtained in the previous step by thier index $i$.
Then scan the input string character by character and construct the following $3$ sets:
\begin{itemize}
\item $S_{01} = \{(s[i], rank(i+1), i) \bigm| i \; mod \; 3 = 0\}$
\item $S_{02} = \{(s[i], s[i+1], rank(i+2), i) \bigm| i \; mod \; 3 = 0\}$
\item $S_{12} = \{(s[i], rank(i+1), i) \bigm| i \; mod \; 3 = 1\}$ $\cup$ $\{(s[i], s[i+1], rank(i+2), i) \bigm| i \; mod \; 3 = 2\}$
\end{itemize}
All these sets can be constructed in linear time. Let sort thef first two sets in lexicographically order (taking the well-known time of $O(\frac{N}{B} log_{\frac{M}{B}} \frac{N}{B})$), and permute the last one by $rank(i)$. Note that the the first two sets are sorted in the same way but stored different information, both useful for the follwing merge step. The last two was already sorted by their position in step 2.

\item \textbf{Merge:}
This step can be made easily using the following comparison function:
\begin{verbatim}
compare(e12, e01, e02) 
  if ( e12.i mod 3 = 1 )  
    return lex-less(e12, e01); /* return true if e1 < e2 in lexicographical order 
  else
    return lex-less(e12, e02);
\end{verbatim}
The merge step can be made in I/O linear time and block transfers.
The output is the array of ranks sorted by suffix's starting position $i$.
\end{enumerate} 
\end{document}